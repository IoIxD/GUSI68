\documentclass[a4paper,dvips]{book}
\usepackage{pslatex}
\usepackage{noweb}
\pagestyle{noweb}
\noweboptions{webnumbering,noidentxref,externalindex,nomargintag,smallcode,longchunks}
\newcommand{\clearemptydoublepage}{\newpage{\pagestyle{empty}\cleardoublepage}}
\usepackage[bookmarks=true,colorlinks=true]{hyperref}
\begin{document}
\frontmatter
\title{
\huge \textsc{Liber} \Huge GVSI\\[3ex]
\large
Annotated Source Code to the\\[1ex]
\Large      
Grand Unified Socket Library\\[5ex]
\large
Copyright \copyright 1992--2000 Matthias Neeracher\\[2ex]
Woven \today\\
for GUSI Version 2.1\\
}
\date{}
\maketitle

\newpage
\pagestyle{empty}
%
% Copyright notice
%
\vspace*{\fill}
\section*{GUSI User License}

My primary objective in distributing \texttt{GUSI} is to have it used as widely as 
possible, while protecting my moral rights of authorship and limiting my 
exposure to liability.

\begin{quotation}
\large Copyright \copyright 1992--2000 Matthias Neeracher
\end{quotation}

Permission is granted to anyone to use this software for any purpose on any
computer system, and to redistribute it freely, subject to the following
restrictions:

\begin{itemize}
\item The author is not responsible for the consequences of use of this software,
  no matter how awful, even if they arise from defects in it.
\item The origin of this software must not be misrepresented, either by explicit
  claim or by omission.
\item You are allowed to distributed modified copies of the software, in source
  and binary form, provided they are marked plainly as altered versions, and 
  are not misrepresented as being the original software.
\end{itemize}

Permission is furthermore granted to print individual copies of the documentation and 
annotated source code of this software, provided that this copyright notice appears in
it unchanged.

While I am giving \texttt{GUSI} away for free, that does not mean that I don't like 
getting appreciation for it. If you want to do something for me beyond your
obligations outlined above, you can

\begin{itemize}
\item Acknowledge the use of  \texttt{GUSI} in the about box of your application and/or in
  your documentation.
\item \href{http://www.iis.ee.ethz.ch/~neeri/macintosh/donations.html}{Send me a CD} as described in 
  my donations web page on
\begin{quotation}
  \verb|http://www.iis.ee.ethz.ch/~neeri/macintosh/donations.html|
\end{quotation}
\end{itemize}

%
% Table of Contents
%
\cleardoublepage
\pdfbookmark{Contents}{TOC}
\tableofcontents

%
% Introduction
%
\clearemptydoublepage
\mainmatter
\raggedright
\setcounter{chapter}{0}%

\chapter{Introduction}

\texttt{GUSI} is a POSIX library for traditional MacOS. Its name, which is an
acronym for \textit{Grand Unified Socket Interface}, hints at its original objective to 
provide access to all the various communications facilities in MacOS through a
common, file descriptor based, interface.

The current incarnation, \texttt{GUSI} 2, represents a 
much-needed rewrite of \texttt{GUSI} and introduces support for POSIX threads.

The most recent version of \texttt{GUSI} may be obtained by anonymous ftp from 
\begin{quotation}
\verb|ftp://sunsite.cnlab-switch.ch/software/platforms/macos/src/mw_c|.
\end{quotation}

There is also a mailing list devoted to discussions about \texttt{GUSI}. You can join the
list by sending a mail to 

\begin{quotation}
\verb|gusi-request@iis.ee.ethz.ch| 
\end{quotation}

whose body consists of the word \texttt{subscribe}.

\section{Design Objectives}

The primary objective of \texttt{GUSI} is to emulate as much as practical of the 
UNIX 98 API for the use in MacOS programs. This is in marked contrast to other
approaches which at first glance might seem similar:

\begin{itemize}
\item \texttt{GUSI} is \textit{not} designed for optimal performance of network communication
	(although \texttt{GUSI} 2 should be faster than \texttt{GUSI} 1 for many purposes). The
	design goal is to make the code as fast as possible without changing the POSIX
	API (e.g., by exposing interrupt level code to the library user).

\item \texttt{GUSI} is \textit{not} designed for maximal compliance with the POSIX API either. The
	goal is to provide as much functionality and as faithful implementation as possible
	while maintaining a strict library approach without writing a separate operating
	system. 
\end{itemize}

While the original \texttt{GUSI} design had to appeal to nebulous "standards" eclectically
drawn from POSIX and BSD APIs, the underlying APIs have now evolved into real 
standards, so \texttt{GUSI} 2 now tries to conform to the \textit{X/Open Single Unix Specification,
Version 2} (also known as \texttt{UNIX 98} as much as possible.

\section{Literate Programs}

This book contains the implementation of \texttt{GUSI 2}. The \texttt{pnoweb}\footnote{\texttt{pnoweb} is 
a reimplementation of Norman Ramsey's \texttt{noweb} system in \texttt{perl}} literate programming system
generates both the source code for \texttt{GUSI 2} and this book from a single source.

The source code consists of interleaved prose commentary and labelled code fragments, arranged in an order 
best suited for presenting to a human reader, rather than one dictated by a compiler. Fragments, referred
to by their labels in angle brackets, consist of source code and references to other fragments. Several 
fragments may have the same name; they will be concatenated in the source code. Fragments work like macros;
a source file is created by expanding one fragment. If its definition refers to other fragments, they are 
themselves expanded, until no fragment references remain. 

\section{How to Read This Book}

This book consists of 5 parts. Part~\ref{part:infra} describes the infrastructure underlying \texttt{GUSI 2}.
Part~\ref{part:POSIX} describes the POSIX-style APIs through which \texttt{GUSI} is accessed, mapping them
onto the interfaces defined in part~\ref{part:infra}. Part~\ref{part:domain} then describes how each communication
domain implements those interfaces. Part~\ref{part:library} presents the code connecting GUSI to the 
vendor specific I/O libraries and console interfaces. Finally, part~\ref{part:build} describes the build
system used for \texttt{GUSI 2}.

\section{Contacting the Author}

I welcome correspondence both about the practical use of \texttt{GUSI 2} and about its presentation in this
book (which I will freely admit still leaves a lot to be desired).

\begin{verbatim}
        Matthias Neeracher 			
        20875 Valley Green Dr. #50			
        Cupertino, CA 95014
	
        e-Mail: <neeracher@mac.com>	
        Fax:   	+1 (408) 514-2605 ext. 0023
\end{verbatim}

%
% Contents
%
\part{Infrastructure}
\label{part:infra}
\input{woven/GUSIDiag.nww}
\input{woven/GUSIBasics.nww}
\input{woven/GUSIContext.nww}
\input{woven/GUSISpecific.nww}
\input{woven/GUSISocket.nww}
\input{woven/GUSIBuffer.nww}
\input{woven/GUSIFactory.nww}
\input{woven/GUSIDevice.nww}
\input{woven/GUSIConfig.nww}
\input{woven/GUSIContextQueue.nww}
\input{woven/GUSIDCon.nww}
\input{woven/GUSITimer.nww}

\clearemptydoublepage
\part{POSIX Wrappers}
\label{part:POSIX}
\input{woven/GUSIDescriptor.nww}
\input{woven/GUSIPOSIX.nww}
\input{woven/GUSIPThread.nww}
\input{woven/GUSISignal.nww}

\clearemptydoublepage
\part{Domain Specific Code}
\label{part:domain}
\input{woven/GUSISocketMixins.nww}
\input{woven/GUSINull.nww}
\input{woven/GUSIPipe.nww}
\input{woven/GUSIInet.nww}
\input{woven/GUSINetDB.nww}
\input{woven/GUSIMTInet.nww }
\input{woven/GUSIMTTcp.nww}
\input{woven/GUSIMTUdp.nww}
\input{woven/GUSIMTNetDB.nww}
\input{woven/GUSIOpenTransport.nww}
\input{woven/GUSIOTInet.nww}
\input{woven/GUSIOTNetDB.nww}
\input{woven/GUSIFSWrappers.nww}
\input{woven/GUSIFileSpec.nww}
\input{woven/GUSIMacFile.nww}
\input{woven/GUSIPPC.nww}

\clearemptydoublepage
\part{Library Specific Code}
\label{part:library}
\input{woven/GUSIMSL.nww}
\input{woven/GUSIMPWStdio.nww}
\input{woven/GUSISfio.nww}
\input{woven/GUSIMPW.nww}
\input{woven/GUSISIOUX.nww}
\input{woven/GUSISIOW.nww}
\input{woven/GUSIForeignThreads.nww}

\clearemptydoublepage
\label{part:build}
\part{The Build System}
\input{woven/Makefile.nww}
%
% Appendixes
%
\clearemptydoublepage
\appendix
\part{Appendixes}
\clearemptydoublepage
\chapter{Index}
\nowebindex
\clearemptydoublepage
\chapter{Names of the Sections}
\nowebchunks
\end{document}
